% ===================================================
% Glossary
% ===================================================
\newglossaryentry{hessian}
{
    name=Hessian-matrise,
    description={En kvadratisk matrise av andre partielle deriverte av en funksjon med flere variabler, som beskriver lokal krumning av funksjonen.},
    symbol={$\nabla^2 f(x)$ eller $H(x)$}
}

\newglossaryentry{feasible-region}
{
    name=tillatt område,
    description={Mengden av alle mulige punkter som tilfredsstiller alle bibetingelser i et optimeringsproblem.},
    symbol={$\mathcal{X}$}
}

\newglossaryentry{convex-function}
{
    name=konveks funksjon,
    description={En funksjon $f$ der linjesegmentet mellom to vilkårlige punkter på funksjonen aldri ligger under funksjonsgrafen. Matematisk: $f(\lambda x + (1-\lambda)y) \leq \lambda f(x) + (1-\lambda)f(y)$ for alle $x, y$ og $\lambda \in [0,1]$.},
    symbol={$f$ er konveks}
}

\newglossaryentry{local-minimum}
{
    name=lokalt minimum,
    description={Et punkt $x^*$ hvor funksjonsverdien er mindre enn eller lik verdiene i en liten omegn rundt punktet. Formelt: Det eksisterer en $\delta > 0$ slik at $f(x^*) \leq f(x)$ for alle $x$ med $\|x - x^*\| < \delta$.},
    symbol={$x^*$ lokalt min}
}

\newglossaryentry{global-minimum}
{
    name=globalt minimum,
    description={Et punkt $x^*$ hvor funksjonsverdien er mindre enn eller lik verdien for alle andre punkter i definisjonsmengden. Formelt: $f(x^*) \leq f(x)$ for alle $x$ i definisjonsmengden.},
    symbol={$x^*$ globalt min}
}

\newglossaryentry{gradient}
{
    name=gradient,
    description={En vektor som gir retningen og størrelsen på den maksimale økningen av en funksjon i et gitt punkt. For en funksjon $f(x)$ er gradienten gitt ved $\nabla f(x) = (\frac{\partial f}{\partial x_1}, \frac{\partial f}{\partial x_2}, \ldots, \frac{\partial f}{\partial x_n})$.},
    symbol={$\nabla f(x)$}
}

\newglossaryentry{lagrange-multiplier}
{
    name=lagrange-multiplikator,
    description={En variabel brukt i lagrange-metoden for å finne ekstremalpunkter til en funksjon underlagt bibetingelser. For et problem med mål $f(x)$ og bibetingelser $g(x) = 0$, danner vi Lagrange-funksjonen $L(x, \lambda) = f(x) - \lambda g(x)$.},
    symbol={$\lambda$}
}

\newglossaryentry{kkt-conditions}
{
    name=KKT-betingelser,
    description={Nødvendige første-ordens betingelser for at en løsning skal være optimal i et optimeringsproblem med bibetingelser, oppkalt etter Karush, Kuhn og Tucker. Inkluderer stasjonaritetsbetingelsen, komplementær slakk, primal og dual gjennomførbarhet.},
    symbol={$\nabla f(x^*) + \sum_{i=1}^m \lambda_i \nabla g_i(x^*) = 0$},
    see={first-order-necessary-conditions}
}
\newglossaryentry{slater-condition}{
    name=slakkbetingelse,
    description={En betingelse i KKT-betingelsene som sier at produktet av Lagrange-multiplikatoren og bibetingelsen må være null. Dette betyr at enten er multiplikatoren null eller så er bibetingelsen lik null.},
    symbol={$g_i(x) \geq 0$},
    see={kkt-conditions}
}

\newglossaryentry{first-order-necessary-conditions}{
    name={First-order necessary conditions},
    description={
        Betingelser som må være oppfylt for at en løsning skal være optimal i et optimeringsproblem. Disse inkluderer gradienten av målfunksjonen og bibetingelsene.
    },
    symbol={$\nabla f(x^*) = 0$}
    see={kkt-conditions},
}

\newglossaryentry{steepest-descent}{
    name={Steepest Descent},
    description={
        En metode for å finne minimum av en funksjon ved å følge den bratteste nedstigningen i gradienten \(d_k = -\nabla f(x_k)\). Dette innebærer å ta et skritt i retning av den negative gradienten for å minimere funksjonen.
    },
    symbol={$d_k = -\nabla f(x_k)$},
    see={gradient},
}




% ==================================================
% Acronyms
% ==================================================
\newacronym{kkt}{KKT}{Karush-Kuhn-Tucker betingelser}

\newacronym{gd}{GD}{Gradient Descent}

\newacronym{ls}{LS}{Line Search}

\newacronym{lp}{LP}{Lineær Programmering}

\newacronym{nlp}{NLP}{Ikke-lineær Programmering}

\newacronym{qp}{QP}{Kvadratisk Programmering}

