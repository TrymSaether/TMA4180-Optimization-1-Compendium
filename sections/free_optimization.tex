\chapter{Fri optimalisering}

Fri optimalisering refererer til problemer uten eksplisitte restriksjoner på variablene.
Målet er å minimere en \textit{glatt objektfunksjon} \( f : \R^d \to \R \) for hele rommet \( \R^d \).

\[
  \min_{\symbf{x} \in \R^d} f(\symbf{x})
\]
der løsningen \( \symbf{x}^* \) tilfredsstiller:
\[
  f(\symbf{x}^*) \leq f(\symbf{x}) \quad \forall \symbf{x} \in \R^d
\]

\section{Egenskaper}

\begin{itemize}
  \item \textbf{Ingen restriksjoner}: Den tillatte mengden (feasible set) er hele \( \R^n \), uten likhets- eller ulikhetsbetingelser.
  \item \textbf{Enklere oppsett}: Det er ikke behov for å håndtere restriksjoner som sikrer at løsningen er gyldig.
  \item \textbf{Fokuserer kun på målfunksjonen}: Algoritmer søker etter punkter \( x \) som reduserer \( f(x) \) direkte, uten å ta hensyn til andre forhold.
\end{itemize}

\subsection{Konvergens}
For å sikre konvergens til en stasjonær løsning i \( x^* \), må følgende betingelser oppfylles:
\begin{enumerate}
  \item \(f(x)\) er kontinuerlig deriverbar i (\(C^1\)).
  \item Gradienten \(\nabla f(x)\) eksisterer og er Lipschitz-kontinuerlig.
  \item \hyperref[def:level_set]{Nivåsettene} \(\mathcal{L}_f(\alpha)\) er begrenset og lukket \hyperref[def:compact_set]{(kompakte)}.
\end{enumerate}
Under disse betingelsene sikres konvergens til en stasjonær løsning \( x^* \) som oppfyller \(\nabla f(x^*) = 0\).
\[
  \lim_{k \to \infty} \|\nabla f(x_k)\| = 0,
\]
der \( x_k \) er iteratene generert av en optimaliseringsalgoritme, og \( x^* \) tilfredsstiller førsteordens nødvendige betingelser:
\[
  \nabla f(x^*) = 0.
\]

