\part{Convex Optimization}

\chapter{Constrained Optimization over Convex Sets}

\section{Problem Formulation}
We consider optimization problems where we minimize a differentiable function $f: \mathbb{R}^n \to \mathbb{R}$ over a convex feasible set $\Omega \subseteq \mathbb{R}^n$:
\begin{mini*}
    {x \in \Omega}{f(x)}{}{}
\end{mini*}

The feasible region $\Omega$ can be defined in several ways:
\begin{itemize}
    \item Explicitly as a convex set, e.g., $\Omega = \{x : \|x\| \leq 1\}$
    \item Through equality constraints: $\Omega = \{x : h(x) = 0\}$ where $h$ is a system of affine functions
    \item Through inequality constraints: $\Omega = \{x : g(x) \leq 0\}$ where $g$ is a system of convex functions
    \item Combinations of the above
\end{itemize}

\section{Feasible Directions}
At a feasible point $x \in \Omega$, a direction $d \in \mathbb{R}^n$ is called:
\begin{itemize}
    \item A \textbf{feasible direction} if there exists $\alpha > 0$ such that $x + td \in \Omega$ for all $t \in (0, \alpha]$.
    \item A \textbf{descent direction} if $\nabla f(x)^T d < 0$.
\end{itemize}

The key insight is that if $d$ is both feasible and a descent direction at $x$, then moving along $d$ from $x$ will reduce the objective value while staying within the feasible region.

\subsection{Cone of Feasible Directions}
The cone of feasible directions at $x \in \Omega$ is defined as:
\[
\mathcal{F}_\Omega(x) = \{d \in \mathbb{R}^n : \exists \alpha > 0 \text{ s.t. } x + td \in \Omega \text{ for all } t \in (0, \alpha] \}
\]

For convex sets, this cone has useful properties that facilitate optimization algorithms.

\section{First-Order Optimality Conditions}

\subsection{Necessary Conditions}
A necessary condition for $x^* \in \Omega$ to be a local minimum is:
\[
\nabla f(x^*)^T d \geq 0 \quad \text{for all feasible directions } d \text{ at } x^*
\]
This generalizes the first-order condition $\nabla f(x^*) = 0$ from unconstrained optimization.

\subsection{Sufficient Conditions}
If $f$ is convex and $\Omega$ is a convex set, then the necessary condition becomes sufficient for global optimality:
\[
\nabla f(x^*)^T (y - x^*) \geq 0 \quad \text{for all } y \in \Omega
\]
This is a generalization of the first-order characterization of convex functions.

\section{Projections on Convex Sets}
For any point $x \in \mathbb{R}^n$ and a closed convex set $\Omega$, the projection of $x$ onto $\Omega$ is defined as:
\[
P_\Omega(x) = \arg\min_{y \in \Omega} \|y - x\|_2
\]

\subsection{Properties of Projections}
\begin{itemize}
    \item The projection onto a closed convex set always exists and is unique.
    \item $x = P_\Omega(x)$ if and only if $x \in \Omega$.
    \item Geometric characterization: $P_\Omega(x)$ is the unique point $y \in \Omega$ such that $(x - y)^T(z - y) \leq 0$ for all $z \in \Omega$.
\end{itemize}

\subsection{Projected Gradient Method}
A natural extension of the gradient descent method for constrained optimization is the projected gradient method:
\[
x_{k+1} = P_\Omega(x_k - \alpha_k \nabla f(x_k))
\]
where $\alpha_k$ is the step size.

The projected gradient method has convergence guarantees similar to gradient descent when applied to convex functions and convex feasible sets.

\begin{algorithm}[H]
    \caption{Projected Gradient Method}
    \begin{algorithmic}[1]
        \State Choose initial point $x_0 \in \Omega$ and stopping tolerance $\epsilon > 0$
        \For{$k = 0, 1, 2, \ldots$}
            \State Compute gradient $\nabla f(x_k)$
            \State Choose step size $\alpha_k > 0$ (e.g., by line search)
            \State Update $x_{k+1} = P_\Omega(x_k - \alpha_k \nabla f(x_k))$
            \If{$\|x_{k+1} - x_k\| < \epsilon$}
                \State \textbf{return} $x_{k+1}$
            \EndIf
        \EndFor
    \end{algorithmic}
\end{algorithm}

This method is particularly useful when projections onto the feasible set can be computed efficiently, as in the case of boxes, balls, or polyhedra.

\chapter{Theory of Convex Optimization}

\section{Slater's Constraint Qualification}

Slater's constraint qualification is a condition that ensures strong duality holds in convex optimization problems.

\begin{definition}{Slater's Condition}{slaters_condition}
    For a convex optimization problem:
    \begin{mini*}
        {x}{f(x)}{}{}
        \addConstraint{g_i(x)}{\leq 0,\quad}{i = 1, \ldots, m}
        \addConstraint{Ax}{= b}{}
    \end{mini*}
    where $f$ and $g_i$ are convex, Slater's condition holds if there exists a point $\bar{x}$ such that:
    \begin{itemize}
        \item $g_i(\bar{x}) < 0$ for all $i = 1, \ldots, m$
        \item $A\bar{x} = b$
    \end{itemize}
    This point $\bar{x}$ is called a strictly feasible point.
\end{definition}

\begin{theorem}{Importance of Slater's Condition}{slater_importance}
    For a convex optimization problem that satisfies Slater's condition:
    \begin{enumerate}
        \item Strong duality holds: the optimal values of the primal and dual problems are equal
        \item If the primal optimal value is finite, the dual optimal value is attained
        \item The KKT conditions are necessary and sufficient for optimality
    \end{enumerate}
\end{theorem}

\section{Farkas' Lemma}

Farkas' Lemma is a fundamental result in convex analysis and optimization that characterizes when a linear inequality is a consequence of a system of linear inequalities.

\begin{lemma}{Farkas' Lemma}{farkas_lemma}
    Let $A \in \mathbb{R}^{m \times n}$ and $c \in \mathbb{R}^n$. Then, exactly one of the following two statements is true:
    \begin{enumerate}
        \item There exists an $x \in \mathbb{R}^n$ such that $Ax \leq 0$ and $c^Tx > 0$.
        \item There exists a $y \in \mathbb{R}^m$ such that $y \geq 0$ and $A^Ty = c$.
    \end{enumerate}
\end{lemma}

Farkas' Lemma is often used to establish the existence of Lagrange multipliers and to derive duality results in optimization.

\section{Lagrangian Function}

The Lagrangian function combines the objective function with constraint functions through Lagrange multipliers.

\begin{definition}{Lagrangian Function}{lagrangian}
    For the constrained optimization problem:
    \begin{mini*}
        {x}{f(x)}{}{}
        \addConstraint{g_i(x)}{\leq 0,\quad}{i = 1, \ldots, m}
        \addConstraint{h_j(x)}{= 0,\quad}{j = 1, \ldots, p}
    \end{mini*}
    The Lagrangian function is defined as:
    \[
    \mathcal{L}(x, \lambda, \mu) = f(x) + \sum_{i=1}^m \lambda_i g_i(x) + \sum_{j=1}^p \mu_j h_j(x)
    \]
    where $\lambda \in \mathbb{R}^m$ with $\lambda \geq 0$ and $\mu \in \mathbb{R}^p$ are the Lagrange multipliers.
\end{definition}

The Lagrangian function plays a central role in constrained optimization, forming the basis for KKT conditions and duality theory.

\section{KKT Conditions for Convex Problems}

The Karush-Kuhn-Tucker (KKT) conditions are necessary and sufficient conditions for optimality in convex optimization problems.

\begin{theorem}{KKT Conditions for Convex Problems}{kkt_convex}
    Consider a convex optimization problem where $f$ and $g_i$ are convex, and $h_j$ are affine. Suppose that Slater's condition is satisfied. Then $x^*$ is an optimal solution if and only if there exist Lagrange multipliers $\lambda^* \in \mathbb{R}^m$ and $\mu^* \in \mathbb{R}^p$ such that:
    \begin{align}
        \nabla f(x^*) + \sum_{i=1}^m \lambda_i^* \nabla g_i(x^*) + \sum_{j=1}^p \mu_j^* \nabla h_j(x^*) &= 0 \tag{Stationarity}\\
        g_i(x^*) &\leq 0, \quad \forall i = 1, \ldots, m \tag{Primal Feasibility}\\
        h_j(x^*) &= 0, \quad \forall j = 1, \ldots, p \tag{Primal Feasibility}\\
        \lambda_i^* &\geq 0, \quad \forall i = 1, \ldots, m \tag{Dual Feasibility}\\
        \lambda_i^* g_i(x^*) &= 0, \quad \forall i = 1, \ldots, m \tag{Complementary Slackness}
    \end{align}
\end{theorem}

\section{Duality in Convex Optimization}

\subsection{Weak and Strong Duality}

\begin{definition}{Dual Function}{dual_function}
    The dual function $g: \mathbb{R}^m \times \mathbb{R}^p \to \mathbb{R}$ is defined as:
    \[
    g(\lambda, \mu) = \inf_{x \in \mathbb{R}^n} \mathcal{L}(x, \lambda, \mu)
    \]
\end{definition}

\begin{definition}{Dual Problem}{dual_problem}
    The dual problem is:
    \begin{maxi*}
        {\lambda \geq 0, \mu}{g(\lambda, \mu)}{}{}
    \end{maxi*}
\end{definition}

\begin{theorem}{Weak Duality}{weak_duality}
    If $x$ is primal feasible and $(\lambda, \mu)$ is dual feasible with $\lambda \geq 0$, then:
    \[
    g(\lambda, \mu) \leq f(x)
    \]
    This means the optimal value of the dual problem is a lower bound on the optimal value of the primal problem.
\end{theorem}

\begin{theorem}{Strong Duality}{strong_duality}
    If the primal problem is convex and Slater's condition holds, then strong duality holds: the optimal values of the primal and dual problems are equal.
\end{theorem}

\subsection{Saddle Points}

\begin{definition}{Saddle Point}{saddle_point}
    A point $(x^*, \lambda^*, \mu^*)$ is a saddle point of the Lagrangian if:
    \[
    \mathcal{L}(x^*, \lambda, \mu) \leq \mathcal{L}(x^*, \lambda^*, \mu^*) \leq \mathcal{L}(x, \lambda^*, \mu^*)
    \]
    for all $x \in \mathbb{R}^n$, $\lambda \geq 0$, and $\mu \in \mathbb{R}^p$.
\end{definition}

\begin{theorem}{Saddle Point Theorem}{saddle_point_theorem}
    For a convex optimization problem, $(x^*, \lambda^*, \mu^*)$ is a saddle point of the Lagrangian if and only if:
    \begin{itemize}
        \item $x^*$ is primal optimal
        \item $(\lambda^*, \mu^*)$ is dual optimal
        \item Strong duality holds
    \end{itemize}
\end{theorem}

\subsection{Lagrangian Duality}

Lagrangian duality provides a systematic way to derive dual problems and establish duality results.

\begin{theorem}{Lagrangian Duality}{lagrangian_duality}
    For a convex optimization problem:
    \[
    \min_{x \in \mathbb{R}^n} f(x) = \min_{x \in \mathbb{R}^n} \max_{\lambda \geq 0, \mu} \mathcal{L}(x, \lambda, \mu)
    \]
    When strong duality holds:
    \[
    \min_{x \in \mathbb{R}^n} \max_{\lambda \geq 0, \mu} \mathcal{L}(x, \lambda, \mu) = \max_{\lambda \geq 0, \mu} \min_{x \in \mathbb{R}^n} \mathcal{L}(x, \lambda, \mu)
    \]
\end{theorem}

\subsection{Legendre-Fenchel Transform}

\begin{definition}{Legendre-Fenchel Transform}{legendre_fenchel}
    The Legendre-Fenchel transform (or convex conjugate) of a function $f: \mathbb{R}^n \to \mathbb{R} \cup \{+\infty\}$ is:
    \[
    f^*(y) = \sup_{x \in \mathbb{R}^n} \{y^Tx - f(x)\}
    \]
\end{definition}

\begin{theorem}{Properties of Legendre-Fenchel Transform}{legendre_fenchel_properties}
    The Legendre-Fenchel transform has several important properties:
    \begin{itemize}
        \item $f^*$ is always convex, even if $f$ is not
        \item If $f$ is convex and lower semicontinuous, then $(f^*)^* = f$
        \item If $f$ is strictly convex and differentiable, then $\nabla f^*(y) = x$ where $y = \nabla f(x)$
    \end{itemize}
\end{theorem}

The Legendre-Fenchel transform provides powerful tools for duality theory and convex analysis, connecting problems through their dual formulations.


\chapter{Linear and Quadratic Programming}

\section{Linear Optimization}
Linear optimization, or linear programming, is one of the most widely used optimization techniques in practice. We consider problems of the form:
\begin{mini*}
    {x\in\R^n}{c^Tx}{}{}
    \addConstraint{Ax}{\leq b}{}
    \addConstraint{Ex}{= d}{}
\end{mini*}
where $A \in \R^{m \times n}$, $E \in \R^{p \times n}$, $c \in \R^n$, $b \in \R^m$, and $d \in \R^p$.

\subsection{Primal and Dual Linear Programs}
For every linear programming problem (primal problem), there exists an associated dual problem:

\begin{maxi*}
    {y \in \R^m, z \in \R^p}{b^Ty + d^Tz}{}{}
    \addConstraint{A^Ty + E^Tz}{= c}{}
    \addConstraint{y}{\geq 0}{}
\end{maxi*}

The relationship between the primal and dual problems provides valuable insights:
\begin{itemize}
    \item Weak duality: The optimal value of the dual problem is always less than or equal to the optimal value of the primal problem.
    \item Strong duality: Under mild conditions, the optimal values of the primal and dual problems are equal.
    \item Complementary slackness: If $x^*$ and $(y^*, z^*)$ are optimal solutions to the primal and dual problems, then $y_i^*(A_ix^*-b_i) = 0$ for all $i$.
\end{itemize}

\section{Quadratic Programming}

\subsection{Quadratic Programs with Equality Constraints}
Quadratic programs with only equality constraints take the form:
\begin{mini*}
    {x\in\R^n}{\frac{1}{2}x^TQx + c^Tx}{}{}
    \addConstraint{Ax}{= b}{}
\end{mini*}
where $Q$ is symmetric and positive semi-definite.

For this problem, the KKT conditions provide a system of linear equations:
\begin{align*}
    Qx + c + A^T\lambda &= 0\\
    Ax &= b
\end{align*}

\subsection{Quadratic Programs with Inequality Constraints}
When inequality constraints are present, the problem becomes:
\begin{mini*}
    {x\in\R^n}{\frac{1}{2}x^TQx + c^Tx}{}{}
    \addConstraint{Ax}{\leq b}{}
    \addConstraint{Ex}{= d}{}
\end{mini*}

The KKT conditions now include complementary slackness:
\begin{align*}
    Qx + c + A^T\lambda + E^T\nu &= 0\\
    Ax &\leq b\\
    Ex &= d\\
    \lambda &\geq 0\\
    \lambda_i(A_ix - b_i) &= 0 \quad \forall i
\end{align*}

\subsection{Active Set Methods}
Active set methods solve quadratic programming problems by iteratively updating an estimate of the active constraint set at the solution. The main steps are:

\begin{algorithm}[H]
    \caption{Active Set Method for QP}
    \begin{algorithmic}[1]
        \State Find an initial feasible point $x_0$
        \State Identify the initially active constraints $\mathcal{A}_0$
        \For{$k = 0, 1, 2, \ldots$}
            \State Solve equality-constrained QP using constraints in $\mathcal{A}_k$
            \State Compute Lagrange multipliers $\lambda_i$ for active constraints
            \If{all $\lambda_i \geq 0$}
                \State Find constraint with most negative $\lambda_i$
                \State Remove this constraint from $\mathcal{A}_k$
            \Else
                \State Find step direction $p_k$
                \State Calculate step length to nearest inactive constraint
                \State Update $x_{k+1} = x_k + \alpha_k p_k$
                \State Update $\mathcal{A}_{k+1}$ with newly active constraint
            \EndIf
            \If{no constraints to add or remove}
                \State \textbf{return} $x_k$
            \EndIf
        \EndFor
    \end{algorithmic}
\end{algorithm}
