\documentclass[tikz,border=5mm]{standalone}
\usetikzlibrary{shapes.geometric, arrows.meta, positioning, fit, backgrounds}

\begin{document}

\begin{tikzpicture}[
node distance=10mm and 0mm,
every node/.style={font=\footnotesize},
decision/.style={diamond, draw, fill=blue!20, text width=3.5cm, align=center, inner sep=2pt},
process/.style={rectangle, draw, fill=green!20, text width=2.5cm, align=center, minimum height=8mm},
io/.style={trapezium, trapezium left angle=70, trapezium right angle=110,
                     draw, fill=yellow!20, text width=3.5cm, align=center, minimum height=8mm},
arrow/.style={-{Stealth[scale=1.0]}, thick}
]

% Start node
\node (start) [io] {Start: Identify optimization problem};

% Unconstrained branch
\node (dec1) [decision, below=of start] {Unconstrained?};
\node (u_grad) [process, below=of dec1] {Check stationarity: \(\nabla f(x^*)=0\)};
\node (u_hess) [decision, below=of u_grad] {Check Hessian \(\nabla^2f(x^*)\)};
\node (u_pd) [process, below left=of u_hess] {PD ⇒ Strict local min};
\node (u_psd) [process, below right=of u_hess] {PSD ⇒ Local min};

% Equality branch
\node (dec2) [decision, right=40mm of dec1] {Equality constrained?};
\node (e_licq) [process, below=of dec2] {Check LICQ: \(\nabla c(x^*)\) linearly independent};
\node (e_kkt) [process, below=of e_licq] {Solve KKT: \(\nabla_xL=0,\;c(x)=0\)};
\node (e_hess) [decision, below=of e_kkt] {Reduced Hessian PD? \(\nabla^2_{xx}L(x^*,\lambda^*)\)};
\node (e_pd) [process, below left=of e_hess] {Yes ⇒ Strict local min};
\node (e_psd) [process, below right=of e_hess] {No ⇒ Local min};

% Inequality branch  
\node (dec3) [decision, right=40mm of dec2] {Inequality constrained?};
\node (i_cq) [process, below=of dec3] {Check LICQ/MFCQ: \(\nabla c(x^*)\) linearly independent};
\node (i_kkt) [process, below=of i_cq] {Solve KKT: \(\nabla_xL=0,\;c\ge0,\;\lambda\ge0,\;\lambda c=0\)};
\node (i_conv) [decision, below=of i_kkt] {Convex + Slater?\\ \(\nabla^2_{xx}L(x^*,\lambda^*)\) PD and \(c(x^*)>0\)};
\node (i_glob) [process, below left=of i_conv] {Yes ⇒ Global min};
\node (i_so) [process, below right=of i_conv] {No ⇒ Check SOSC};

% Arrows
\draw[arrow] (start) -- (dec1);

% Unconstrained arrows
\draw[arrow] (dec1) -- node[left]{yes} (u_grad);
\draw[arrow] (u_grad) -- (u_hess);
\draw[arrow] (u_hess) -- node[left]{yes} (u_pd);
\draw[arrow] (u_hess) -- node[right]{no} (u_psd);

% Equality arrows
\draw[arrow] (dec1) -- node[above]{no} (dec2);
\draw[arrow] (dec2) -- node[left]{yes} (e_licq);
\draw[arrow] (e_licq) -- (e_kkt);
\draw[arrow] (e_kkt) -- (e_hess);
\draw[arrow] (e_hess) -- node[left]{yes} (e_pd);
\draw[arrow] (e_hess) -- node[right]{no} (e_psd);

% Inequality arrows
\draw[arrow] (dec2) -- node[above]{no} (dec3);
\draw[arrow] (dec3) -- node[left]{yes} (i_cq);
\draw[arrow] (i_cq) -- (i_kkt);
\draw[arrow] (i_kkt) -- (i_conv);
\draw[arrow] (i_conv) -- node[left]{yes} (i_glob);
\draw[arrow] (i_conv) -- node[right]{no} (i_so);

% Legend
\node [rectangle, draw, fill=white, below=20mm of i_so, text width=12cm, align=left, font=\scriptsize] {
\textbf{Legend:}\\
PD = Positive Definite\\
PSD = Positive Semidefinite\\
LICQ = Linear Independence Constraint Qualification\\
MFCQ = Mangasarian-Fromovitz Constraint Qualification\\
SOSC = Second Order Sufficient Conditions\\
KKT = Karush-Kuhn-Tucker conditions
};

\end{tikzpicture}
\end{document}

