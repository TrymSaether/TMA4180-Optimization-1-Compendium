\part{Betinget Optimering}

\chapter{Introduksjon til Betinget Optimering}

Betinget optimering er en sentral del av optimeringsteori som handler om å finne optimale verdier for en funksjon når variablene må tilfredsstille spesifikke betingelser eller begrensninger.

\section{Problemformulering}

\chapter{Teori for Betinget Optimering}

\section{Tangenskjegler}

For generelle (muligens ikke-konvekse) mengder spiller tangenskjeglen en avgjørende rolle i optimalitetsbetingelser.

\begin{definition}{Tangenskjegle}{tangent_cone}
    For en mengde \( \Omega \subset \mathbb{R}^n \) og et punkt \( x \in \Omega \), er tangenskjeglen \( T_\Omega(x) \) definert som:
    \[
        T_\Omega(x) = \left\{ d \in \mathbb{R}^n : \exists \{x_k\} \subset \Omega, \{t_k\} \subset \mathbb{R}^+, t_k \downarrow 0, \lim_{k\to\infty} \frac{x_k - x}{t_k} = d \right\}
    \]
\end{definition}

Tangenskjeglen generaliserer konseptet med mulige retninger til ikke-konvekse settinger. For konvekse mengder sammenfaller tangenskjeglen med kjeglen av mulige retninger.

\section{Lineær Uavhengighetsbetingelse (LICQ)}

I optimalisering med både likhets- og ulikhetsbetingelser:
\begin{mini*}
    {x}{f(x)}{}{}
    \addConstraint{g_i(x)}{\leq 0,\quad}{i \in \mathcal{I}}
    \addConstraint{h_j(x)}{= 0,\quad}{j \in \mathcal{E}}
\end{mini*}

møter vi ofte situasjoner der visse kvalifikasjonsbetingelser må være oppfylt for å sikre veloppførte optimalitetsbetingelser.

\begin{definition}{Aktiv Mengde}{active_set}
    For et mulig punkt \( x \), er den aktive mengden \( \mathcal{A}(x) \):
    \[
        \mathcal{A}(x) = \{i \in \mathcal{I} : g_i(x) = 0\} \cup \mathcal{E}
    \]
\end{definition}

\begin{definition}{Lineær Uavhengighetsbetingelse (LICQ)}{licq}
    For et mulig punkt \( x \), er LICQ oppfylt hvis settet av gradienter for aktive betingelser:
    \[
        \{\nabla g_i(x) : i \in \mathcal{A}(x) \cap \mathcal{I}\} \cup \{\nabla h_j(x) : j \in \mathcal{E}\}
    \]
    er lineært uavhengig.
\end{definition}

LICQ sikrer at betingelsene er veldefinerte og at Lagrange-multiplikatorene for den optimale løsningen er unike.

\section{KKT-betingelser for Ikke-konvekse Problemer}

\begin{theorem}{KKT Nødvendige Betingelser}{kkt_necessary}
    La \( x^* \) være et lokalt minimum for optimaliseringsproblemet med begrensninger, og anta at LICQ er oppfylt i \( x^* \). Da eksisterer det unike Lagrange-multiplikatorer \( \lambda^* \in \mathbb{R}^{|\mathcal{I}|} \) og \( \mu^* \in \mathbb{R}^{|\mathcal{E}|} \) slik at:
    \begin{align}
        \nabla f(x^*) + \sum_{i \in \mathcal{I}} \lambda_i^* \nabla g_i(x^*) + \sum_{j \in \mathcal{E}} \mu_j^* \nabla h_j(x^*) & = 0 \tag{Stasjonaritet}                                           \\
        g_i(x^*)                                                                                                                & \leq 0, \quad \forall i \in \mathcal{I} \tag{Primal Feasibilitet} \\
        h_j(x^*)                                                                                                                & = 0, \quad \forall j \in \mathcal{E} \tag{Primal Feasibilitet}    \\
        \lambda_i^*                                                                                                             & \geq 0, \quad \forall i \in \mathcal{I} \tag{Dual Feasibilitet}   \\
        \lambda_i^* g_i(x^*)                                                                                                    & = 0, \quad \forall i \in \mathcal{I} \tag{Komplementær Slakkhet}
    \end{align}
\end{theorem}

I motsetning til det konvekse tilfellet, er disse betingelsene kun nødvendige, men ikke tilstrekkelige for optimalitet i ikke-konvekse problemer.

\section{Andreordens Nødvendige og Tilstrekkelige Betingelser}

For ikke-konvekse problemer gir andreordens betingelser ytterligere informasjon om naturen til kritiske punkter.

\begin{definition}{Kritisk Kjegle}{critical_cone}
    For et punkt \( x^* \) som oppfyller KKT-betingelsene med multiplikatorer \( (\lambda^*, \mu^*) \), er den kritiske kjeglen \( \mathcal{C}(x^*, \lambda^*) \):
    \[
        \mathcal{C}(x^*, \lambda^*) = \{d \in T_\Omega(x^*) : \nabla g_i(x^*)^T d = 0 \text{ for alle } i \in \mathcal{A}(x^*) \text{ med } \lambda_i^* > 0\}
    \]
\end{definition}

\begin{theorem}{Andreordens Nødvendig Betingelse}{second_order_necessary}
    La \( x^* \) være et lokalt minimum som oppfyller LICQ, med tilhørende Lagrange-multiplikatorer \( (\lambda^*, \mu^*) \). Da gjelder:
    \[
        d^T \nabla^2_{xx} \mathcal{L}(x^*, \lambda^*, \mu^*) d \geq 0 \quad \forall d \in \mathcal{C}(x^*, \lambda^*)
    \]
    der \( \mathcal{L} \) er Lagrange-funksjonen.
\end{theorem}

\begin{theorem}{Andreordens Tilstrekkelig Betingelse}{second_order_sufficient}
    La \( x^* \) være et punkt som oppfyller KKT-betingelsene med multiplikatorer \( (\lambda^*, \mu^*) \). Hvis:
    \[
        d^T \nabla^2_{xx} \mathcal{L}(x^*, \lambda^*, \mu^*) d > 0 \quad \forall d \in \mathcal{C}(x^*, \lambda^*) \setminus \{0\}
    \]
    da er \( x^* \) et strengt lokalt minimum.
\end{theorem}

\chapter{Metoder for betinget optimering}

\section{Straffemetoder}
En vanlig tilnærming for å håndtere betingelser i optimaliseringsproblemer er å legge til \emph{straffetermer} i målfunksjonen. Straffetermer gir store kostnader når betingelsene brytes, og leder løsningen gradvis mot det tillatte området.

\begin{definition}{Straffemetode (Penalty Method)}{penalty_method}
    Betrakt det generelle problemet
    \[
        \min_{x\in\mathbb{R}^n} f(x)
        \quad
        \text{s. t. }
        g_i(x)\le0,\;i\in\mathcal I,
        \;
        h_j(x)=0,\;j\in\mathcal E.
    \]
    En straffefunksjon med parameter \(r>0\) kan defineres som
    \[
        P(x,r)
        = f(x)
        + r\sum_{i\in\mathcal I}\bigl[\max\{0,\,g_i(x)\}\bigr]^2
        + r\sum_{j\in\mathcal E}\bigl[h_j(x)\bigr]^2.
    \]
    Minimering av \(P(x,r)\) for voksende \(r\) tvinger iteratene mot det egentlige begrensningssettet.
\end{definition}

\subsection{Kvadratiske straffemetoder}
En spesiell variant er å straffe alle betingelser med kvadrater, også de som allerede er tilfredsstilt:
\[
    P_{\mathrm{quad}}(x,r)
    = f(x)
    + r\sum_{i\in\mathcal I} g_i(x)^2
    + r\sum_{j\in\mathcal E} h_j(x)^2.
\]
Her økes \(r\) typisk etter hvert iterasjonstrinn for å redusere bruddene på betingelsene.

\subsection{Barrieremetoder}
Barrieremetoder introduserer en term som blir uendelig når man nærmer seg grensene for de ugyldige områdene. En populær barriere er den logaritmiske barrieren:

\begin{definition}{Logaritmisk barrieremetode}{barrier_method}
    For ulikhetsbetingelser \(g_i(x)\le0\) defineres
    \[
        B(x,\mu)
        = f(x)
        - \mu\sum_{i\in\mathcal I}\ln\bigl(-g_i(x)\bigr),
        \quad \mu>0.
    \]
    Minimering av \(B(x,\mu)\) med synkende \(\mu\) følger den såkalte \emph{sentralveien} mot en KKT-løsning.
\end{definition}

I praksis velges en sekvens \(\{\mu_k\}\downarrow0\). For hvert \(\mu_k\) løses:
\[
    x^{(k)} = \arg\min_x B(x,\mu_k),
\]
og \(x^{(k)}\) initierer neste subproblem.

\subsection{Augmenterte Lagrange-metoder}
Augmenterte Lagrange-metoder kombinerer fordeler fra Lagrange-funksjonen og kvadratiske straffetermer:

\begin{definition}{Augmentert Lagrange-funksjon}{augmented_lagrangian}
    La \(\lambda\in\mathbb R^{|\mathcal E|}\) være Lagrange-multiplikatorer og \(r>0\) en straffeparameter. Den augmenterte Lagrange-funksjonen er
    \[
        \mathcal L_A(x,\lambda,r)
        = f(x)
        + \sum_{j\in\mathcal E}\lambda_j\,h_j(x)
        + \tfrac r2\sum_{j\in\mathcal E} h_j(x)^2.
    \]
\end{definition}

Algoritmisk oppdateres \(\lambda\) og \(r\) etter hvert subproblem:
\[
    x^{(k)} = \arg\min_x \mathcal L_A(x,\lambda^{(k)},r^{(k)}),
    \quad
    \lambda^{(k+1)} = \lambda^{(k)} + r^{(k)}\,h(x^{(k)}),
    \quad
    r^{(k+1)} \ge r^{(k)}.
\]

\section{Pareto}
\label{sec:pareto}

I multiobjektiv optimalisering ønsker man å minimere (eller maksimere) flere mål samtidig:
\[
    \min_{x\in\mathbb{R}^n} \;F(x)
    \quad\text{med}\quad
    F(x) = \bigl(f_1(x),f_2(x),\dots,f_k(x)\bigr).
\]
Ofte finnes det ingen entydig løsning som er best for alle delmål, og man ser etter Pareto--optimale løsninger.

\begin{definition}{Pareto--dominans}{pareto_dominance}
La \(x,y\in\mathbb R^n\). Vi sier at \(x\) \emph{dominerer} \(y\) (skriver \(x \prec y\)) hvis
\[
    f_i(x) \le f_i(y)\quad\forall\,i=1,\dots,k,
    \quad
    \exists\,j:\;f_j(x) < f_j(y).
\]
\end{definition}

\begin{definition}{Pareto--optimalitet}{pareto_optimality}
Et punkt \(x^*\) er Pareto--optimalt hvis det ikke finnes noe \(y\) slik at \(y \prec x^*\). Mængden av alle Pareto--optimale punkter kalles Pareto--mængden, og bildet av denne i målfunksjonsrommet kalles Pareto--fronten:
\[
    \mathcal P = \{x\in\mathbb R^n : \nexists\,y\text{ med }y\prec x\},
    \quad
    \mathcal{PF} = F(\mathcal P).
\]
\end{definition}

\subsection{Vektet Sum Metoden}
Vektet sum metoden er en enkel tilnærming for å håndtere flere mål ved å kombinere dem til ett mål. Dette gjøres ved å tilordne vekter til hvert mål og minimere den vektede summen:
\begin{definition}{Vektet Sum Metode}{weighted_sum_method}
    Betrakt problemet
    \[
        \min_{x\in\mathbb{R}^n} \;F(x)
        \quad\text{med}\quad
        F(x) = \bigl(f_1(x),f_2(x),\dots,f_k(x)\bigr).
    \]
    Den vektede sum metoden løser:
    \[
        \min_{x\in\mathbb{R}^n} \;\sum_{i=1}^k w_i f_i(x)
        \quad\text{med}\quad
        w_i > 0,\;i=1,\dots,k.
    \]
    Her er \(w_i\) vektene som bestemmer hvor mye hvert mål bidrar til den totale kostnaden.
    Vektene kan justeres for å utforske forskjellige deler av Pareto-fronten.
\end{definition}


